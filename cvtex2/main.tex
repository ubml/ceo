% -- Encoding UTF-8 without BOM
% -- XeLaTeX => PDF (BIBER)

\documentclass[]{cv-style}          
% Add 'print' as an option into the square bracket to remove colours from this template for printing. 
                                    
% Add 'espanol' as an option into the square bracket to change the date format of the Last Updated Text

\sethyphenation[variant=british]{english}{} % Agregue palabras entre {} para evitar que se corten

\begin{document}

\header{Justino}{Alanoca}           % Your name
\lastupdated

%------------- 
%	BARRA LATERAL - En el lado, cada nueva línea fuerza un salto de línea
%---------------------´

\begin{aside}
%
\section{contacto}
Av juliaca 2021
Puno, Perú
~
 (+51)951806081
~
jalanoca@gmail.com
%
\section{idiomas}
Aymara - jakearu
English - read and write
Español - fluído
%
\section{programación}
{\color{red} $\varheartsuit$} R
VFOX, PHP, JAVA, VBA, SQL
\LaTeX{}
\section{javascript, css:}
Bootstrap 3.3.7, Morris Chart,
Jvectormap, Date Ranger Picker,
DataTables, iCheck
~
javascript:
jQuery Knob Chart  
Chart JS, InputMask  
SweetAlert, jQuery Number  
AdminLTE App  
~
css:
Font Awesome, lonicons, Theme Style AdminLTE, Skins AdminLTE  
\section{SO}
Windows(95, 98, xp, 8, 10, server)
Linux(deepin, ubunto, centos, )
Unix(SCO, Digital)
%
\end{aside}

%----------------------- 
%	 SECTOR HABILIDADES
%----------------------- 

\section{habilidades}
  \vspace{-0.2cm}
Creativo,Innovador, Disruptivo, Trabajo bajo presión 
%------------------------- 
%	 SECCION EXPERIENCIA
%------------------------ 
\section{experiencia}
\begin{entrylist}
%------------------------------------------------
\entry
  {2017--Hoy}
  {UBML}
  {Puno, Puno}
  {\jobtitle{Promotor}\\
Descripción del trabajo.Descripción del trabajo.Descripción del trabajo.Descripción del trabajo.Descripción del trabajo.Descripción del trabajo.}
%------------------------------------------------
\entry
  {2011--2016}
  {Universidad NAcional Autónoma de Huanta}
  {Huanta, Ayacucho}
  {\jobtitle{Jefe de la Oficina de Informática}\\
  Descripción del trabajo.Descripción del trabajo.Descripción del trabajo.Descripción del trabajo.. \\
  Detalle de logros:
  \begin{itemize}
    \item Logro 1. Logro 1. Logro 1.
  \end{itemize}}
%------------------------------------------------
\entry
  {2008--2011}
  {MEF}
  {Lima, Lima}
  {\jobtitle{Consultor}\\
Descripción del trabajo.Descripción del trabajo.Descripción del trabajo.\\
  Detalle de logros:
  \begin{itemize}
    \item Ya no vale tanto como antes las experiencias y responsabilidades pasadas. Quieren saber que has logrado en el presente: en tu curriculum usá palabras que terminen en “é”: “desarrollé”, “inicié”, “aumenté”. Por otro lado, tenés que estar preparado para hablar con ellos, evitá las afirmaciones vagas. Deciles números, cifras concretas, y si los tenés, contales tus premios o reconocimientos, demostrá confianza y seguridad en lo que decís.
  \end{itemize}}
%------------------------------------------------
\entry
  {2007--2008}
  {SUNAT}
  {Huancayo, Junín}
  {\jobtitle{Analista}\\
Descripción del trabajo.Descripción del trabajo.Descripción del trabajo.\\}
%------------------------------------------------
\end{entrylist}

%------------------------------ 
%	SECCION EDUCACION 
%-------------------------------- 
\section{educación}
\begin{entrylist}
%------------------------------------------------
\entry
{2010--2011}
{M.Sc. {\normalfont en Administración [Grado]}}
{UANCV}
{\vspace{-0.3cm}}
%------------------------------------------------
\entry
{2004--2009}
{B.Eng. {\normalfont Ingeniero de Sistemas [Grado]}}
{UANCV}
{(Especialización en planificación y presupuesto ...)}
%------------------------------------------------
\end{entrylist}

%-------------------- 
%	 SECCION CUALIDADES
%-------------------- 

\section{cualidades}
\begin{entrylist}
%------------------------------------------------
\entry
{2013}
{Cualificación}
{SERVIR}
{\vspace{-0.3cm}}
%------------------------------------------------
\entry
{2011}
{Cualificación}
{SUNAT}
{\vspace{-0.3cm}}
%------------------------------------------------
\end{entrylist}

%------------------ 
%	 SECCION RECONOCIMIENTOS
%------------------ 

\section{reconocimientos}
\begin{entrylist}
%------------------------------------------------
\entry
{2014}
{Detalle del premio}
{Institución}
{Descripción del premio, Descripción del premio,Descripción del premio,Descripción del premio,Descripción del premio,Descripción del premio }
%------------------------------------------------
\end{entrylist}

%-------------------- 
%	 SECCION INTERES
%--------------------- 

\section{interes}
  \vspace{-0.2cm}

\textbf{profesional:} interes profesional  1, interes profesional  2 y interes profesional  3. \textbf{personal:} interes personal 1,  interes personal 2,  interes personal 3  y interes personal 4.

%---------------------------------
\end{document}
